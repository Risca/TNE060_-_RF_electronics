\documentclass[report.tex]{subfiles}
\begin{document}
\subsection{Discussion}
Overall, the simulation results in this chapter represents the theory very well, but some points are worth discussing
\subsubsection{Microstrip Parameters}
The results from the simulation given in table~\ref{table: Lab1 Simulated Microstrip parameters} only differs one decimal from its corresponding analytical values given in section \ref{subsubsec: Microstrip_Parameters}. The differences probably depends on more realistic and improved implementations of the parameter computation. The formulas used in Section \ref{subsubsec: Microstrip_Parameters} are a compromise between exactness and hand calculation easiness and the ADS implementations are probably not that easy to calculate by hand.
\subsubsection{Signal Propagation Delay}
The propagation speed calculated in Section \ref{subsubsec:Lab 1 Signal propagation delay} gives a speed that is very reasonable for signal transmission in a copper transmission line. With more data over a wider range of transmission line lengths it should be possible to get a higher resolution of $\Delta t_p$.
\subsubsection{Reflection Coefficient}
The results given in Figure \ref{fig:Lab 1 Voltage recorded at the output.} in Section \ref{subsubsec:Lab1 ref coeff} are consistent with Table \ref{table: RC and SWR} in Section \ref{subsubsec: Lab1 Phase Shift}. One can see that when the load have a smaller resistance than the transmission line destructive interference will occur due to the phase shift between the incident and reflected wave. The opposite occurs when the load is larger than the transmission line and the incident and reflected wave will sum up instead (positive interference).
\subsubsection{Transmission Line as Load}
By using different transmission line lengths it is possible to implement different reactance values where one of the most applicable is the $\sfrac{\lambda}{4}$ transmission line. This can for example be used, as seen in Figure \ref{fig:Lab 1 Voltage recorded at the output.}b), to remove DC bias when connected to ground. This is due to the fact that the  $\sfrac{\lambda}{4}$ transmission line acts as an infinite inductance and will therefore only DC will be connected to ground.
\end{document}
