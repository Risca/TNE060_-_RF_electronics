\documentclass[12pt]{article}

\usepackage[utf8]{inputenc}
\usepackage[swedish,english]{babel}
\usepackage{amsmath}
\usepackage{subfiles}
%\usepackage{graphicx}
\usepackage{subfig}
\usepackage{xfrac}
\usepackage{float}
\usepackage{multirow}
\usepackage[bottom,perpage,symbol*]{footmisc}
\DefineFNsymbols*{lamportnostar}[math]{\dagger\ddagger\S\P\|{\dagger\dagger}{\ddagger\ddagger}}
\setfnsymbol{lamportnostar}

\DeclareGraphicsExtensions{.eps,.pdf,.png,.jpg}
\graphicspath{{graphs/}}

\newcommand{\superscript}[1]{\ensuremath{^{\textrm{#1}}} }
\newcommand{\subscript}[1]{\ensuremath{_{\textrm{#1}}} }
\newcommand{\st}[0]{\superscript{st}}
\newcommand{\nd}[0]{\superscript{nd}}
\newcommand{\rd}[0]{\superscript{rd}}

\title{RF Electronics}

\foreignlanguage{swedish}{
\author{
        Patrik Dahlström \\
        MSc. in Electronics Design Engineering\\
            \and
        Daniel Josefsson\\
        MSc. in Electronics Design Engineering
}
}
\date{\today}

\begin{document}
\maketitle

\vspace{\fill}

\begin{abstract}
This report shows how signals propagate in RF networks, how improperly matched inputs and outputs influence the efficiency of a system, how to match RF networks using simple matching networks, and how Agilent ADS can be used to develop and test these matching networks.

It also gives an introduction on how to develop band-pass filters using both lumped components and distributed networks.
\end{abstract}

\pagebreak
\tableofcontents

\subfile{introduction}

\subfile{lab1.tex}

\subfile{lab2.tex}

\subfile{lab3.tex}

\subfile{conclusion.tex}

\end{document}
