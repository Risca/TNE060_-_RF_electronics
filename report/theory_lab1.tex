\documentclass[report.tex]{subfiles}
\begin{document}
\subsection{Theoretical background}\label{sec: Lab1 Theoretical Background}
To get a picture of how well the simulations of a microstrip in ADS are, the parameters of a microstrip  were analysed analytically for comparison. This was done by solving three basic problems.

\subsubsection{Microstrip Parameters}\label{subsubsec: Microstrip_Parameters}
The microstrip that was analysed was defined in problem 2.13 in [1]:

``It is desired to construct a 50 $\Omega$ microstrip line. The relative dielectric constant is 2.23 and the board height is h = 787 $\mu \text{m.}$ Find the width, wavelength and effective dielectric constant when the thickness of the copper trace is negligible. Assume an operating frequency of 1 GHz."

\begin{table}[h]
    \centering
    \caption{Summary of microstrip parameters.}
    \begin{tabular}{c | c | c | c | c | c | c}
        $Z_{in} [\Omega]$ & $\epsilon_r $ & $h [\mu \text{m}]$ & $w [\text{mm}]$ & $\lambda [\text{mm}]$ & $\epsilon_{eff}$ & $f_0 [\text{GHz}]$\\
        \hline
         50 & 2.23 & 787 & ? & ? & ? & 1
    \end{tabular}
    \label{table: Lab1 Microstrip parameters}
\end{table}

By using $Z_{f} = \sqrt{\mu_{0} \epsilon_{0}} = 376.8\:\Omega$ together with the formula for the effective dielectric constant;
\begin{equation}
	\epsilon_{eff} = \dfrac{\epsilon_{r} + 1}{2} + \dfrac{\epsilon_{r} - 1}{2} \left(1 + {\dfrac{12 h}{w}} \right)^{-1/2} \text ,
\end{equation}
the formula for characteristic impedance
\begin{equation}
	Z_0 = \dfrac{Z_f}{\sqrt{\epsilon_{eff}}\left(1.393 + \dfrac{w}{h} + \dfrac{2}{3} \text{ln}\left(\dfrac{w}{h} + 1.444 \right)\right)}
\end{equation}
where the width-height ratio is computed with
\begin{equation}
	\dfrac{w}{h} = \dfrac{2}{\pi}\left(B - 1 - \text{ln}\left(2B - 1 \right) + \dfrac{\epsilon_r - 1}{2 \epsilon_r}\left(\text{ln}\left(B -1\right) + 0.39 - \dfrac{0.61}{\epsilon_r}\right)\right)
\end{equation}
and where B is given by
\begin{equation}
\dfrac{Z_f \pi}{2Z_0\sqrt{\epsilon_r}}
\end{equation}
and the formula for wavelength
\begin{equation}
	\lambda = \dfrac{c}{f\sqrt{\epsilon_{eff}}}.
\end{equation}
the sought parameters can be solved for. This gives $w = 2.4 \text{mm}$, $\lambda = 218 \text{mm}$ and $\epsilon_{eff} = 1.89 \text{.}$

\subsubsection{Transmission Line Phase Shift}\label{subsubsec: Lab1 Phase Shift}
When designing a microstrip that has a length that differs from the wavelength a mismatch will occur. So, what phase shift will occur if the microstrip given in \ref{subsubsec: Microstrip_Parameters} has a length of 15 cm?

Since the wavelength is longer that the microstrip the phase shift will be less than one period and therefore can the phase shift be calculated as
\begin{equation}
	\text{ps} = \dfrac{l}{\lambda} 360 \:^\circ
\end{equation}
and this gives ps = 247.8 $^\circ$.
\subsubsection{Reflection Coefficient and Standing Wave Ratio}\label{subsec:Lab1 RC and SWR}
Two other properties of a microstrip are the reflection coefficient, $\Gamma_0$ and the (voltage) standing wave ratio, (V)SWR. These two parameters are computed with
\begin{equation}
	\Gamma_0 = \dfrac{Z_L - Z_0}{Z_L + Z_0}
\end{equation}
and
\begin{equation}
	SWR = \dfrac{1 + \left|\Gamma_0\right|}{1 - \left|\Gamma_0\right|}\text{.}
\end{equation}

Due to the fact that the characteristic impedance of the transmission line and the load have the same impedance $Z_0 = Z_L = 50\:\Omega$ the circuit is matched. This means that all of the energy is transmitted to the load and no reflective wave will occur. This is also given by the value of the reflection constant, $\Gamma_0 = 0$. Since $\Gamma_0$ is zero, $SWR = 1$ and this is the ideal case and shows that the circuit is matched. Table~\ref{table: RC and SWR} below shows the reflection coefficient and standing wave ratio for some given loads.

\begin{table}[h]
    \centering
    \caption{$\Gamma_0$ and SWR for different loads when $Z_0= 50\: \Omega$.}
    \begin{tabular}{c | c c c c c}
         $Z_L [\Omega]$ & 0 & 25 & 50 & 100 & 1E6 \\
         \hline
         $\Gamma_0$ & -1 & -$\sfrac{1}{3}$ & 0 & $\sfrac{1}{3}$ & 1 \\
         SWR & $\infty$ & 2 & 0 & 2 & $\infty$
    \end{tabular}\label{table: RC and SWR}
\end{table}

Negative values for $\Gamma_0$ in Table \ref{table: RC and SWR} represents a 180 $^\circ$ phase shift for the reflected wave compared to the incident.
\end{document}
	
