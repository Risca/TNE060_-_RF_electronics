\documentclass[report.tex]{subfiles}
\begin{document}
\subsection{Theoretical background}
When designing a pass- or stop-band filter, a low-pass filter is used as basis and then frequency shifted and impedance transformed. 

Theoretically designing a maximally flat 2\nd order lumped component band-pass filter using the parameters in table~\ref{table:filter specs} is done by first designing a low-pass filter with cut-off frequency equal to the desired center frequency ($\omega_c=\omega_0$). The center frequency is calculated from eq.~\ref{eqn:center frequency}.

\begin{equation}
    \omega_0 = \sqrt{\omega_U \cdot \omega_L} \approx 2.44~GHz
    \label{eqn:center frequency}
\end{equation}

A flat 2\nd order low-pass filter have the following filter constants (from table 5-2 in R. Ludwig).

\begin{equation*}
    g_0=g_3=1, g_1=g_2=1.4142
\end{equation*}

This low-pass filter can be realized with a series inductor and a shunt capacitor, both with a normalized value of $g_1~F$ and $g_2~H$ respectively. To convert this low-pass filter to a band-pass filter, Table 5-5 in R. Ludwig is considered. A series capacitor is appended to a series inductor, and a shunt capacitor is connected to an inductor in parallel.



\end{document}